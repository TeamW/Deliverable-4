\subsection{Roles}
Gordon Reid was responsible for maintaining and configuring the software tools
used by the team. For example, he set up and maintained the various GitHub
repositories, including making sure the repositories were never cluttered
with unnecessary files. He was also the main proof reader for all
deliverables.

ADD ROLES FOR OTHERS.

\subsection{Authority}

The group has decided to delegate authority by role. Roles are volunteered for
on a temporary basis, allowing both flexibility and persisitence, as the
member or group decides. Roles can be rotated or adjusted to allow for
circumstances such as deadlines or illness. This approach was chosen to play
to the individual members' strengths and to facilite group cohesion, as none
of us can claim to be `senior' to the others.

Decisions will primarily be made on a democratic voting method where any
major changes are brought forward to the whole group, discussed, then voted
on. For minor decisions, such as role specific decisions, these can be made
by the individual it soley affects.

With the democratic voting method with prior discussion disputes should be
rare. Majority vote rules and should be accepted by all members of the team.

\subsection{Communication}

The group will be using three main methods of communication: a communal
Facebook group, bi-weekly face-to-face meetings in the third year lab,
and GitHub.

Facebook will provide us the tools to post messages for all to see, and the
ability to have real time chat with members currently online.

The face to face meetings will allows us to take part in pair programming,
group discussion, and general progress report.

GitHub will be used to communicate source code changes via the commit log. In
addition to this there is also an `Issues' section in which known bugs can be
viewed, added to, or commented on. There is also the option for creating
milestones with associated tickets to provide an overview of overall project
progress.

The methods were also chosen to allow information flow at any time, and from
any location. Facebook allows instant communication of important information
since the group are all equipped with smartphones. The extensive logging
facilities of GitHub let us leave detailed information on any changes to the
current system and documentation. GitHub also provides numerous graphical
representations of quantitative data. This can help visualise overall
progress through the project.

The group decided early on that regular face-to-face meetings were crucial
in the broader aspects of design, since any misunderstandings or
miscommunication at this level could lead to massive problems further down
the line.

\subsection{Information Management}

Every file of source code and every deliverable is kept in a GitHub repository
which simulataneously acts as central version control and backup. Every
team member has full access to every repository associated with the project.

Separate from the shared documentation, each team member will be keeping a
weekly blog on the University of Glasgow's Mahara website. This will be
inaccessible by other members of the team as it plays no part in the overall
project development.

Any physical documentation created, for instance meeting minutes, will be
backed up electronically either via photographs being taken of the pages or
by typing the documentation up and uploading to the appropriate repository.

\subsection{Organisational Risks}

The loose structure of the group does have various associated risks. The
lack of concrete structure means the team has to stay on task, as they
have no boss to drive them should they slack off. The weekly meetings
have been implemented to try and offset this, but it is possible that this
may not be as effective as hoped.

The changing structure of the roles could lead to confusion over who is
doing what from week to week. We plan to clarify the weeks tasks at the
end of each meeting to counter this issue.

The current structure has no assurance that the person who has volunteered
for the tasks has the technical ability to do so. This may be irrelevant
for this project as it is part of an educational course however should still
be discussed and planned for. The open nature of the team should naturally
resolve this as help is only a Facebook post, or GitHub issue away.

MORE NEEDED
