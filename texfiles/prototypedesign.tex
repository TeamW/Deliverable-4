The design of the prototype was driven by the simplicity of the scope and the bash language. Once we agreed on
what we wished to learn from the prototype and used this to decide the scope, we were able to draw up a simple
set of text-based menus that demonstrated the functionality and workflow such that the clients could have a 
meaningful look at how we understood the requirements. We very quickly focussed on the idea that the system has 
multiple types of user and so designed the system such that this is addressed immediately, directing the user to
options that apply to them. In this way we show the way the eventual implementation will separate the users 
functionality without having to actually implement the login system, which was not feasable. The submission, 
accepting and viewing of adverts were so key to the system requirements that they were implemented to a 
more tangible degree, using a simple file creation/storage system to allow the dynamic creation of very simple
textual adverts. We did this to allow the user to create adverts and see their progress through the system, rather 
than simply show a demo advert which may have skipped over small issues. The other category of features 
within the scope of the system were those that were to be included, but not made functional. These were 
implemented in a very simple manner, comprising of basic print statements that did nothing beyond this.