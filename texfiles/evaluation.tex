Prototype Evaluation.

The prototype successfully demonstrates the main functionality of the final build of the system operating with small, non-critical additions to be made in order to finialise it, but does not accurately illustrate how the final implementation of the system will look as it will contain a GUI and more user friendly with buttons to navigate and security for logging into the different user types amongst other small necessary additions.

The program due to being minimalist in nature because it's what the client requested limits the flexibility; there are no rich text formats, colours, tabbing, images or anything other than simple text for the company to write their adverts with. There is no need to amend or add on more functionality into the prototype in the future because it would be gold plating and unnecessary in this case.

The usability of the program is incredibly easy and very user friendly. It's a text based program in which there are numerous menu's available depending on what type of user you are - student/course coordinator/company. Navigating this simple program is a case of selecting an option (numbered) in each menu. The main functionality of say a particular company would be to add adverts (& amend), this requires numerous sections to appear where the company must fill in these sections with as much detail as possible for the student - e.g. locations, how to contact, requirements, payment, time etc. Simplistic tasks to follow before its fully added and thus also complies with the proper formatting of adverts.

The user interface is as mentioned above incredibly simplistic being text based. Selecting an option such as "3) Student", on the starting menu which asks which type of user you are will open the Student menu where other options, numbered, will appear and can be selected by entering the number of the option. Eventually you come across an option that doesn't navigate to other menus but instead performs a function such as, for students, the viewing of all the accepted advertisements then at the top you'll see the advert as created by the specific company. There is no clutter on this interface just the options that are needed and navigating from the main menu to each of the users respective main "functionalities" takes no more than 3 menu navigations, all the while still maintaining the option of going back to previous stages for ease of use.

The prototype currently has no security features e.g. integration with the university system to allow users to log in with their GUID is outside the scope of this and thus currently any user can access each individual section just by navigating from the menu's with the right numbered option. Being a prototype this isn't a problem because the final system would have some sort of system that requires you to log in (though not linked to the GUID's).

The system is currently persistent over multiple restarts of the system as the number of adverts up for evaluation, rejected and accepted are all stored along with the adverts themselves in an external file as soon as they're manipulated in some way. This keeps the system simplistic and reliable for any users but a final system would require some sort of updating database online and when started, takes the information from that database to display.
 
The feedback given to the team by Tim after the prototype demonstration was completed was mostly positive, small additions were to be added on request such as different sections while creating an advert to follow the proper formatting but mainly the flow of the program was easy to understand, simple to navigate and the main functionality is as requested.



